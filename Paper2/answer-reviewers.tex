\documentclass[10pt]{article}

\usepackage{fullpage}
\usepackage{url}

\begin{document}

\noindent
{\Large\bf ANSWERS TO THE REVIEWERS}\\~\\

We thank the  reviewers for their interest in our work and for their helpful comments that will greatly improve  the manuscript. We have  tried  to do  our  best to  respond to  the  points  raised. The reviewers have brought  up  some  good  points  and  we  appreciate  the  opportunity  to  clarify  our research objectives and results. As  indicated  below,  we  have  checked  all  the  general  and  specific  comments  provided  by  the reviewers and have made necessary changes accordingly to their indications.\\

\noindent{\large\bf Anonymous reviewers \#1}\\

\noindent
{\em 1) The usage workflow is not that clear. For instance how optimisations can be selected ? Is there a list of such optimisations or is it the user
  to design the optimisation sets. In Section V.A how different values are selected? Is there any way for the framework to guide the user towards
  optimisations or to give feedback on what is working, what is not working, etc. The authors mention MAQAO in the conclusion... Overall, this part is
  really fuzzy.}\\

\noindent
{\em 2) Listing 3 should be explained in details or removed. Not every reader is a specialist of vectorisation}\\

\noindent
{\em 3) section V.A page 8. You state that you have an unexpected results that should be investigate. A journal paper is not a conference paper. I think you should investigate and publish what you have found in ***this*** paper.} \\

\noindent
{\em 4) Section V.C. is not easy to follow. It is hard to understand where the problems you faced came from : from the kernels, from the design of BOAST, from a bug in the implementation, etc. What kind of conclusion should be taken from your experience in the usage of BOAST? Such discussion should help the reader in doing better science. The writing of this section does not go in that direction. What can be generalised from your experience? what are your advices?}\\

\noindent
{\em 5) in Section V.C why did you not generate the CUDA version as well and compare it with the hand-tuned existing version? For the revised version, I need to see that experiment added to paper!}\\ 


\end{document}
%%%

Reviewer: 1

Comments to the Author

This paper present a framework, called BOAST,  for developing computational kernels in a portable efficient and productive way. It provides a DSL and a set of tools to generate different source code with different optimisations and to autotune the code in order to produce  an optimised version for the target architecture.

I think this is a good work with several examples showing how the framework works and how it yields to interesting results.

I have several remarks in order to make this paper better.









