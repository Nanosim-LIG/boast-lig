\documentclass[review]{elsarticle}

\usepackage{lineno,hyperref}
\modulolinenumbers[5]

\journal{Journal of Parallel and Distributed Computing}

%%%%%%%%%%%%%%%%%%%%%%%
%% Elsevier bibliography styles
%%%%%%%%%%%%%%%%%%%%%%%
%% To change the style, put a % in front of the second line of the current style and
%% remove the % from the second line of the style you would like to use.
%%%%%%%%%%%%%%%%%%%%%%%

%% Numbered
%\bibliographystyle{model1-num-names}

%% Numbered without titles
%\bibliographystyle{model1a-num-names}

%% Harvard
%\bibliographystyle{model2-names.bst}\biboptions{authoryear}

%% Vancouver numbered
%\usepackage{numcompress}\bibliographystyle{model3-num-names}

%% Vancouver name/year
%\usepackage{numcompress}\bibliographystyle{model4-names}\biboptions{authoryear}

%% APA style
%\bibliographystyle{model5-names}\biboptions{authoryear}

%% AMA style
%\usepackage{numcompress}\bibliographystyle{model6-num-names}

%% `Elsevier LaTeX' style
\bibliographystyle{elsarticle-num}
%%%%%%%%%%%%%%%%%%%%%%%

\begin{document}

\begin{frontmatter}

\title{BOAST: a Metaprogramming Framework to Produce Portable and Efficient Computing Kernels for HPC Applications}

%% or include affiliations in footnotes:
\author[mymainaddress]{Brice Videau}
\ead{brice.videau@imag.fr}

\address[mymainaddress]{LIG/CNRS}

\begin{abstract}
The Abstract.
\end{abstract}

\begin{keyword}
Code Generation \sep Portability \sep High Performance Computing \sep
Autotuning \sep Non Regression Testing
\end{keyword}

\end{frontmatter}

\linenumbers

\section{Introduction}

Porting and tuning HPC applications to new platforms is tedious and costly
in term of human resources. Portability efforts are often lost when
migrating to a new architecture, or code lose maintainability because
several versions of the code coexist, usually with a lot of duplication.
Thus productivity of porting and tuning efforts is low as a huge fraction
of those developments are never used after the platform they were intended
for is decommissioned.

\section{Background and Motivation}

  \subsection{Evolution of HPC Architectures}

Evolution of HPC Architectures is rapid and also diverse: in the last 5
years no less than 6 architectures have been number one in the Top500:
\begin{itemize}
\item Intel Processor + Xeon Phi (Tianhe-2)
\item AMD Processor + NVIDIA GPU (Titan)
\item IBM BlueGene/Q (Sequoia)
\item Fujitsu SPARC64 (K computer)
\item Intel Processor + NVIDIA GPU (Tianhe-1)
\item AMD Processor (Jaguar)
\end{itemize}
Being able to efficiently use those architectures on such a small
time-frame is challenging.

The race to exascale is not going to simplify the environment. All of the
above architectures can be considered. For instance European FP7 project Deep
considers using Accelerators (XEON Phi) while the European FP7 project
Mont-Blanc considers using low-power embedded processor with integrated
GPU. Network architectures are also part of the architecture and can be
very diverse.

  \subsection{HPC Applications}

Developed by physicist, chemist, meteorologist. Usually in FORTRAN for
historical and performance reason. Codes can be quite huge (several
thousands lines of code) with lots of functionalities. Nonetheless they are
usually based on computing kernels. Computing kernels are resource
intensive well defined parts of a program.

  \subsection{One extreme case: the Mont-Blanc prototype}


  \subsection{State of the Art: Application Autotuning}


\section{Use Cases}

  \subsection{BigDFT}


  \subsection{SPECFEM3D}


\section{Using Code Generation in Application Autotuning}

  \subsection{General Principles}


  \subsection{Our Contribution: BOAST}
  \cite{videau2013boast}
     + dumper pour tests de non régression


\section{Creating an Auto-Tuned Convolution Library for BigDFT using BOAST}



\section{Porting SPECFEM3D to OpenCL using BOAST}



\section{Related Work}



\section{Conclusion and Future Works}


\section*{References}

\bibliography{BOAST}

\end{document}
